\documentclass[11pt,openright,a4paper]{report}
%%
%% This document template assumes you will use pdflatex.  If you are using
%% latex and dvipdfm to translate to pdf, insert dvipdfm into the options.
%%

%%
%% Package includes to provide the basic style
%%
\usepackage{harvard}    % Uses harvard style referencing
\usepackage{graphicx}   % Permits import of various graphics formats
\usepackage{hyperref}   % Provides hyperlinks to sections automatically
\usepackage{pdflscape}  % Provides landscape mode for end code listings
\usepackage{multicol}   % Provides ability to split output into columns
\usepackage{listings}   % Provides styled code listings


%%
%% Set some page size changes from the standard article class
%%
\usepackage{calc}
\setlength{\parskip}{6pt}
\setlength{\parindent}{0pt}
\addtolength{\hoffset}{-0.5cm}
\addtolength{\textwidth}{2.5cm}


%%
%% Format definitions for the style
%%
\bibliographystyle{agsm}  %{alpha}
\citationstyle{dcu}
\pagestyle{headings}
\fussy


%%
%% Definitions to provide layout in the dissertation title pages
%%
\newenvironment{spaced}[1]
  {\begin{minipage}[c]{\textwidth}\vspace{#1}}
  {\end{minipage}}


\newenvironment{centrespaced}[2]
  {\begin{center}\begin{minipage}[c]{#1}\vspace{#2}}
  {\end{minipage}\end{center}}


\newcommand{\declaration}[2]{
  \thispagestyle{empty}
  \begin{spaced}{4em}
    \begin{center}
      \LARGE\textbf{#1}
    \end{center}
  \end{spaced}
  \begin{spaced}{3em}
    \begin{center}
      Submitted by: #2
    \end{center}
  \end{spaced}
  \begin{spaced}{5em}
    \section*{COPYRIGHT}

    Attention is drawn to the fact that copyright of this dissertation rests
    with its author. The Intellectual Property Rights of the products
    produced as part of the project belong to the author unless otherwise specified
    below, in accordance with the University of Bath's policy on intellectual property 
   (see http://www.bath.ac.uk/ordinances/22.pdf).

    This copy of the dissertation has been supplied on condition that anyone
    who consults it is understood to recognise that its copyright rests with its
    author and that no quotation from the dissertation and no information
    derived from it may be published without the prior written consent of
    the author.

    \section*{Declaration}
    This dissertation is submitted to the University of Bath in accordance
    with the requirements of the degree of Bachelor of Science in the
    Department of Computer Science. No portion of the work in this dissertation
    has been submitted in support of an application for any other degree
    or qualification of this or any other university or institution of learning.
    Except where specifically acknowledged, it is the work of the author.
  \end{spaced}

  \begin{spaced}{5em}
    Signed:
  \end{spaced}
  }


\newcommand{\consultation}[1]{%
\thispagestyle{empty}
\begin{centrespaced}{0.8\textwidth}{0.4\textheight}
\ifnum #1 = 0
This dissertation may be made available for consultation within the
University Library and may be photocopied or lent to other libraries
for the purposes of consultation.
\else
This dissertation may not be consulted, photocopied or lent to other
libraries without the permission of the author for #1 
\ifnum #1 = 1
year
\else
years
\fi
from the date of submission of the dissertation.
\fi
\vspace{4em}

Signed:
\end{centrespaced}
}

%%
%% END OF DEFINITIONS
%%

    %% These are the includes required for the doc 


\title{Your Dissertation title}
\author{Your name}
\date{Bachelor of Science in Computer Science with Honours\\The University of Bath\\Month YEAR}


\begin{document}


% Set this to the language you want to use in your code listings (if any)
\lstset{language=Java,breaklines,breakatwhitespace,basicstyle=\small}


\setcounter{page}{0}
\pagenumbering{roman}


\maketitle
\newpage


% Set this to the number of years consultation prohibition, or 0 if no limit
\consultation{0}
\newpage


\declaration{Dissertation title}{Your name}
\newpage


\abstract
Your abstract should appear here.  An abstract is a short
paragraph describing the aims of the project, what was
achieved and what contributions it has made.
\newpage


\tableofcontents
\newpage
\listoffigures
\newpage
\listoftables
\newpage


\chapter*{Acknowledgements}
Add any acknowledgements here.
\newpage


\setcounter{page}{1}
\pagenumbering{arabic}



\chapter{Introduction}
%% Uncomment this to include a separate tex file wih the introduction contents
%\include{introduction.tex}

This is the introductory chapter.

\section{Example Section}
Like all chapters, it will have a number of sections

\subsection{Example Subsection}
\ldots and sub-sections

\subsubsection{Example sub-subsection}
\ldots and sub-subsections.

\begin{table}[htb]
\begin{center}
\caption{An example table}
\label{Example-Table}
\begin{tabular}{|l|l|}
\hline
Items & Values \\
\hline
\hline
Item 1 & Value 1 \\
Item 2 & Value 2 \\
\hline
\end{tabular}
\end{center}
\end{table}

\section[short section title]{Another section}
Another section, just for good measure.
You can referene a table, figure or equation using \verb|\ref|, just
like this reference to table \ref{Example-Table}.

\section{Example lists}

\subsection{Enumerated}

\begin{enumerate}
\item Example enumerated list
  \begin{itemize}
  \item a nested enumerated list item
  \end{itemize}
\item Second item in the list
\end{enumerate}

\subsection{Itemized}

\begin{itemize}
\item Example itemized list
  \begin{itemize}
  \item a nested itemized list item
  \end{itemize}
\item Second item in the list
\end{itemize}

\subsection{Description}

\begin{description}
\item[Item 1] Example description list
\item[Item 2] Second item in the list
\end{description}


\chapter{Literature Survey}
%% Uncomment this to include a separate tex file wih the introduction contents
%\include{litsurvey.tex}
This is the chapter for your Literature Survey.

You will wish to cite authors like \cite{latex} or \cite{btxdoc}.  Alternate
commands are used to cite \citeasnoun{latex} as a noun, or cite
\possessivecite{latex} work possessively, or add text to the citation, 
\citeaffixed{latex}{e.g.}.

If these citations do not compile correctly, ensure you have the Harvard
package installed.  You can pick up the Harvard package in the zip file
of the dissertation template files you downloaded.


%%
%% NOTE: Replace the following with chapters that are appropriate for your
%%       style of project.  It is unlikely these will fit your project perfectly.
%%

\chapter{Requirements}
If you are doing a primarily software development project, this is the
chapter in which you review the requirements decisions and
critique the requirements process.


\chapter{Design}
This is the chapter in which you review your design decisions at various
levels and critique the design process.


\chapter{Implementation and Testing}
This is the chapter in which you review the implementation and testing
decisions and issues, and critique these processes.

Code can be output inline using \verb@\lstinline|some code|@.  For example,
this code is inline: \lstinline|public static int example = 0;|  (I have
used the character \verb@|@ as a delimiter, but any non-reserved character
not in the code text can be used.)

Code snippets can be output using the \verb|\begin{lstlisting} ... \end{lstlisting}|
environment with the code given in the environment.  For
example, consider listing \ref{Example-Code}, below.

\begin{lstlisting}[breaklines,breakatwhitespace,caption={Example code},label=Example-Code]
public static void main() {

  System.out.println("Hello World");

}
\end{lstlisting}

Code listings are produced using the package ``Listings''.  This has many
useful options, so have a look at the package documentation for further
ideas.


\chapter{Results}
This is the chapter in which you review the outcomes, and
critique the outcomes process.  You may include user evaluation here
too.


%%
%% Now we are back to the standard project contents that you should include
%%

\chapter{Conclusions}
%% Uncomment this to include a separate tex file wih the conclusion contents
%\include{conclusion.tex}

This is the chapter in which you review the major achievements in the
light of your original objectives, critique the process, critique your
own learning and identify possible future work.


\bibliography{ExampleBibFile}


\appendix

%%
%% Use the appendix for major chunks of detailed work, such as these. Tailor
%% these to your own requirements
%%

\chapter{Design Diagrams}

\chapter{User Documentation}

\chapter{Raw results output}

\chapter{Code}

%%
%% NOTE that for this to typeset correctly, ensure you use the pdflatex
%%      command in preference to the latex command.  If you do not have
%%      the pdflatex command, you will need to remove the landscape and
%%      multicols tags and just make do with single column listing output
%%

\begin{landscape}
\begin{multicols}{2}
\section{File: yourCodeFile.java}
\lstinputlisting[basicstyle=\scriptsize]{yourCodeFile.java}
\end{multicols}
\end{landscape}

\end{document}
